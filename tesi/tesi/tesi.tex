        %%******************************************%%
        %%                                          %%
        %%        Modello di tesi di laurea         %%
        %%            di Andrea Giraldin            %%
        %%                                          %%
        %%             2 novembre 2012              %%
        %%                                          %%
        %%******************************************%%


% I seguenti commenti speciali impostano:
% 1. 
% 2. PDFLaTeX come motore di composizione;
% 3. tesi.tex come documento principale;
% 4. il controllo ortografico italiano per l'editor.

% !TEX encoding = UTF-8
% !TEX TS-program = pdflatex
% !TEX root = tesi.tex
% !TEX spellcheck = it-IT

\documentclass[10pt,                    % corpo del font principale
               a4paper,                 % carta A4
               twoside,                 % impagina per fronte-retro
               openright,               % inizio capitoli a destra
               english,                 
               english,                 
               ]{book}    

\usepackage[T1]{fontenc}                % codifica dei font:
% NOTA BENE! richiede una distribuzione *completa* di LaTeX

\usepackage[utf8]{inputenc}             % codifica di input; anche [latin1] va bene
                                        % NOTA BENE! va accordata con le preferenze dell'editor

%**************************************************************
% Importazione package
%************************************************************** 

%\usepackage{amsmath,amssymb,amsthm}    % matematica

\usepackage[english]{babel}    % per scrivere in italiano e in inglese;
                                        % l'ultima lingua (l'italiano) risulta predefinita

\usepackage{bookmark}                   % segnalibri

\usepackage{caption}                    % didascalie

\usepackage{chngpage,calc}              % centra il frontespizio

\usepackage{csquotes}                   % gestisce automaticamente i caratteri (")

\usepackage{emptypage}                  % pagine vuote senza testatina e piede di pagina

\usepackage{epigraph}					% per epigrafi

\usepackage{eurosym}                    % simbolo dell'euro



%\usepackage{indentfirst}               % rientra il primo paragrafo di ogni sezione

\usepackage{graphicx}                   % immagini

\usepackage{hyperref}                   % collegamenti ipertestuali

\usepackage[binding=5mm]{layaureo}      % margini ottimizzati per l'A4; rilegatura di 5 mm

\usepackage{listings, lstautogobble}                   % codici

\usepackage{microtype}                  % microtipografia

\usepackage{mparhack,fixltx2e,relsize}  % finezze tipografiche

\usepackage{nameref}                    % visualizza nome dei riferimenti                                      

\usepackage[font=small]{quoting}        % citazioni

\usepackage{subfig}                     % sottofigure, sottotabelle

\usepackage[english]{varioref}          % riferimenti completi della pagina

\usepackage[dvipsnames, table, x11names]{xcolor}         % colori

\usepackage{booktabs}                   % tabelle                                       
\usepackage{tabularx}                   % tabelle di larghezza prefissata                                    
\usepackage{longtable}                  % tabelle su più pagine                                        
%\usepackage{ltxtable}                   % tabelle su più pagine e adattabili in larghezza

\usepackage[toc, acronym]{glossaries}   % glossario
                                        % per includerlo nel documento bisogna:
                                        % 1. compilare una prima volta tesi.tex;
                                        % 2. eseguire: makeindex -s tesi.ist -t tesi.glg -o tesi.gls tesi.glo
                                        % 3. eseguire: makeindex -s tesi.ist -t tesi.alg -o tesi.acr tesi.acn
                                        % 4. compilare due volte tesi.tex.

\usepackage[backend=biber,style=ieee,hyperref]{biblatex}
                                        % eccellente pacchetto per la bibliografia; 
                                        % produce uno stile di citazione autore-anno; 
                                        % lo stile "numeric-comp" produce riferimenti numerici
                                        % per includerlo nel documento bisogna:
                                        % 1. compilare una prima volta tesi.tex;
                                        % 2. eseguire: biber tesi
                                        % 3. compilare ancora tesi.tex.

%**************************************************************
% Pacchetti  e comandi Jordan
%**************************************************************

\usepackage{float}
\usepackage{makecell}
\usepackage{enumitem}
\usepackage{tcolorbox}
\usepackage{booktabs}
%\usepackage[space]{grffile}

\usepackage{ltablex}
\usepackage{tabu}
\usepackage{amssymb}% http://ctan.org/pkg/amssymb
\usepackage{pifont}% http://ctan.org/pkg/pifont
\usepackage{algorithm} 
\usepackage{algpseudocode}

\raggedbottom


\definecolor{I}{HTML}{0172CE} %old:0076FF %CornflowerBlue!50
\definecolor{P}{HTML}{FFFFFF}
\definecolor{D}{HTML}{CCE6FF}
\newcolumntype{Y}{>{\centering\arraybackslash}X} % colonna X centrata per tabularx
\renewcommand{\tabularxcolumn}[1]{>{\small}m{#1}}
\newcommand\VRule[1][\arrayrulewidth]{\color{white} \vrule width 1pt}

\newcommand\green[1]{\textcolor{ForestGreen}{#1}} %colore testo 
\newcommand\red[1]{\textcolor{Red}{#1}}
\newcommand\super[0]{\green{S}}
\newcommand\nonsuper[0]{\red{NS}}
\newcommand\impl[0]{\green{I}}
\newcommand\nonimpl[0]{\red{NI}}

\newcommand\greencheck[0]{\green{\ding{51}}}
\newcommand\yellowcheck[0]{\textcolor{YellowOrange}{\ding{51}}}
\newcommand\redx[0]{\red{\ding{55}}}

\newcommand\imgref[1]{\hyperref[#1]{figure \ref{#1}}}
\newcommand\imgrefcap[1]{\hyperref[#1]{Figure \ref{#1}}}

%\lstset{% Add other global options here
%	%basicstyle=\small\sffamily,
%	escapechar=\&,	% char to escape out of listings and back to LaTeX
%	autogobble=true,
%	tabsize=4,
%	numbers=left, numberstyle=\tiny, stepnumber=2, numbersep=5pt
%}

\algnewcommand\algorithmicforeach{\textbf{for each}}
\algdef{S}[FOR]{ForEach}[1]{\algorithmicforeach\ #1\ \algorithmicdo}


\input{tesi-config}                     % file con le impostazioni personali

\begin{document}
%**************************************************************
% Materiale iniziale
%**************************************************************
\frontmatter
%% !TEX encoding = UTF-8
% !TEX TS-program = pdflatex
% !TEX root = ../tesi.tex

%**************************************************************
% Frontespizio 
%**************************************************************
\begin{titlepage}

\begin{center}

\begin{LARGE}
\textbf{\myUni}\\
\end{LARGE}

\vspace{10pt}

\begin{Large}
\textsc{\myDepartment}\\
\end{Large}

\vspace{10pt}

\begin{large}
\textsc{\myFaculty}\\
\end{large}

\vspace{30pt}
\begin{figure}[htbp]
\begin{center}
\includegraphics[height=6cm]{logo-unipd}
\end{center}
\end{figure}
\vspace{30pt} 

\begin{LARGE}
\begin{center}
\textbf{\myTitle}\\
\end{center}
\end{LARGE}

\vspace{10pt} 

\begin{large}
\textsl{\myDegree}\\
\end{large}

\vspace{40pt} 

\begin{large}
\begin{flushleft}
\textit{Supervisor}\\ 
\vspace{5pt} 
\profTitle \myProf \\
\vspace{10pt}
\textit{Co-supervisor}\\
\vspace{5pt} 
\corrTitle \myCorr \\ 
\end{flushleft}

\vspace{0pt} 

\begin{flushright}
\textit{Author}\\ 
\vspace{5pt} 
\myName
\end{flushright}
\end{large}

\vspace{1pt}

\line(1, 0){338} \\
\begin{normalsize}
\textsc{Academic Year \myAA}
\end{normalsize}

\end{center}
\end{titlepage} 
%\input{inizio-fine/colophon}
%\input{inizio-fine/dedica}
%% !TEX encoding = UTF-8
% !TEX TS-program = pdflatex
% !TEX root = ../tesi.tex

%**************************************************************
% Sommario
%**************************************************************
\cleardoublepage
\phantomsection
\pdfbookmark{Abstract}{Abstract}
\begingroup
\let\clearpage\relax
\let\cleardoublepage\relax
\let\cleardoublepage\relax

\chapter*{Abstract}

The increasingly pervasive use of technology in the automotive industry and urban environment requires the development of broadcasting algorithms to deliver messages across vehicular ad-hoc networks (VANETs). These kinds of networks are created spontaneously and rely on vehicle-to-vehicle communication, without the need of any infrastructure or prior network topology knowledge by nodes. It is foreseeable that, in the near future, VANETs could be exploited in order to run heterogeneous applications, ranging from leisure-oriented functionalities such as video streaming and gaming, to more serious data exchanging services to monitor traffic congestion. One important application consists in emergency message distribution, where message delivery, timeliness and other life-safety related metrics are paramount. Whereas most of related work in this field is focused on one-dimensional topologies (i.e., car platooning in highways), this thesis consists in the reimplementation and redesign for two and three-dimensional scenarios of the RObust and Fast Forwarding algorithm (ROFF) and the Fast-Broadcast algorithm to compare them in various urban scenarios of increasing complexity. Considering urban scenarios, the Obstacle Model will be employed to take into account the shadowing effects of buildings on signal propagation. Moreover, this thesis proposes a Smart Junction extension for both algorithms, SJ-Fast-Broadcast and SJ-ROFF, which increase message delivery ratios by exploiting the presence of vehicles within road junctions in scenarios where the shadowing effects of obstacles is significant.

\endgroup			

\vfill


%% !TEX encoding = UTF-8
% !TEX TS-program = pdflatex
% !TEX root = ../tesi.tex

%**************************************************************
% Ringraziamenti
%**************************************************************
\cleardoublepage
\phantomsection
\pdfbookmark{Ringraziamenti}{ringraziamenti}

%\begin{flushright}{
%	\slshape    
%	``Life is really simple, but we insist on making it complicated''} \\ 
%	\medskip
%    --- Confucius
%\end{flushright}
%
%
%\bigskip

\begingroup
\let\clearpage\relax
\let\cleardoublepage\relax
\let\cleardoublepage\relax

\chapter*{Ringraziamenti}

\noindent \textit{Innanzitutto, vorrei esprimere la mia gratitudine al Prof. Tullio Vardanega, relatore della mia tesi, per l'aiuto fornitomi durante la stesura del documento.}\\

\noindent \textit{Ringrazio IBC, in particolare Denis Corà e Stefano Gesuato, per il supporto durante il periodo di stage.}\\

\noindent \textit{Desidero infine ringraziare la mia famiglia per il sostegno che mi ha fornito in questi anni e Giulia, Giovanni P. e Giovanni D. per il lavoro svolto durante la realizzazione del progetto di ingegneria del software.}\\
\bigskip

\noindent\textit{\myLocation, \myTime}
\hfill \myName

\endgroup


\input{inizio-fine/indici}
\cleardoublepage

%**************************************************************
% Materiale principale
%**************************************************************
\mainmatter
% !TEX encoding = UTF-8
% !TEX TS-program = pdflatex
% !TEX root = ../tesi.tex

\chapter{Introduction}

	\section{Emergency Message Dissemination and broadcasting protocols}
		\label{sec:emd}
		Emergency Message Dissemination (EMD) is a fundamental application in \acrshort{vaneta} to prevent traffic accidents, thereby reducing death and injury rates. Such task can be executed by the VANET itself by turning it into an infrastructure-less self-organizing network, where the dissemination is carried out by specific protocols. 
		
		
		Since traffic information, especially emergency data, has a broadcast-oriented nature (i.e. it is of public interest), it is more appropriate to disseminate it using broadcasting routing scheme rather than unicast or multicast ones. \cite{5989903}
		%		suddivisione broadcasting protocols
		
		This choice leads to some advantages, such as:
		\begin{itemize}
			\item the fact that vehicles do not need to know the destination address and how to calculate a route towards it;
			\item a greater coverage of vehicles interested in the information, useful also in lossy scenarios, especially when paired with controlled redundancy schemes;
			\item a greater efficiency in bandwidth usage.
		\end{itemize}
		
		The idea behind existing algorithms consists in designating the next forwarder in the multi-hop chain from the source of the alert to the target region where the sensitive data has to be delivered. Ideally, the farthest vehicle from a previous forwarder in the dissemination direction should be given priority when designating the next forwarder. However, due to unreliable wireless channel, the designation of farthest vehicle can fail and interrupt the message dissemination. Due to this, next forwarder designation keeps into consideration also other vehicles (called potential forwarder candidates, PFCs) which have received an Alert Message. The PFCs participate in contention to elect the farthest forwarder candidate (FFC) who will continue disseminating the message.
		
		
		In order to carry out the forwarder designation process, the main idea consists in differentiating waiting times (WT) of PFCs. Each PFC should select a waiting time ranging from 0 to a predefined upper bound (PUB). To guarantee the correct designation of the farthest vehicle as forwarder, PFCs choose their waiting time inversely proportional to the distance between the PFC and the previous forwarder. This way other candidates can detect the transmission from the FFC and suppress their transmission.
		
		Advancements in research on Emergency Message Dissemination has lead to the development of a number of broadcasting protocols. However, as identified by Panichpapiboon et al.\cite{5989903}, most of them belong to one of two main categories:
		\begin{itemize}
			\item Single-hop Broadcasting Protocols, in which no flooding is employed. Instead, vehicles periodically select and broadcast only a subset of the packets it has received;
			\item Multi-hop Broadcasting Protocols, in which packets are transmitted through the network via flooding by some of the neighbors of the source. It is of utmost importance to reduce the number of redundant transmission in order not to waste bandwidth and saturate the channel.
		\end{itemize}
		
			\subsection{Single-hop Broadcasting Protocols}
		
		Vehicles employing Single-Hop Broadcasting protocols will not flood received packets immediately through the network. Instead, vehicles use information from packets to update their database and periodically rebroadcast only a fraction of that information. The two variables these kind of protocols can work on to aim for network efficiency are:
		\begin{itemize}
			\item \textit{Broadcast Interval}, i.e. the amount of time between retransmissions, which should keep into consideration both freshness of information and potential redundancy in transmissions;
			\item \textit{Relevancy of information} to broadcast: as stated before, only relevant information (i.e. a subset of all the information) should be broadcast.
		\end{itemize}
		
		Single-Hop protocols can be further subdivided into two categories:
		\begin{enumerate}
			
			\item Fixed Broadcast Interval, which keep the Broadcast Interval fixed. Some examples are:
			\begin{itemize}
				\renewcommand\labelitemi{--}
				
				\item \textit{TrafficInfo}\cite{4621303}, a protocol in which vehicles record, among other information, travel times on road segments (identified by an ID) and keep them on their on-board database. Vehicles periodically exchange information about the learned travel times based on the relevance of such information. The relevance is calculated using a ranking algorithm which uses the current position of the vehicle and the current time (i.e. relevance decreases with distance and time), broadcasting only the $k$  most important information. 
				
				\item \textit{TrafficView}\cite{1263039}, in which vehicles exchange information about speed and position and record it in their database. Data about different vehicles is then aggregated into a single record using one of two aggregation algorithms:
				\begin{itemize}
					\item the \textit{ratio-based} algorithm, which assigns an aggregation ratio to each portion of a road: the more important the road is, the higher the aggregation ratio will be, increasing the accuracy of the information of that area;
					\item the \textit{cost-based} algorithm, an algorithm which keeps into consideration the cost of aggregating different records. The aggregation cost is defined as the loss of accuracy the aggregation will bring about.
				\end{itemize} 
			\end{itemize}
			\item Adaptive Broadcast Interval, which adapt the Broadcast Interval based on dynamic information. Some examples are:
			\begin{itemize}
				\renewcommand\labelitemi{--}
				
				\item \textit{Collision Ratio Control Protocol (CRCP)}\cite{4357748}, a scheme according to which vehicles exchange information about location, speed and road ID. The Broadcast Interval is dynamically controlled based on the amount of detected collisions and bandwidth efficiency: the protocol tries to maintain the number of collisions under a certain threshold by doubling the Broadcast Interval every time the threshold is exceeded. Otherwise, the Broadcast Interval is decreased by one second when the bandwidth efficiency decreases too much.
				
				Moreover, the authors propose three different methods for selecting the data to be transmitted:
				\begin{itemize}
					\item \textit{Random Selection}: a vehicle selects a random information in its database and broadcasts it;
					\item \textit{Vicinity Priority Selection}: vehicles give priority to information of nearby areas;
					\item \textit{Vicinity Priority Selection with Queries}: similar to Vicinity Priority Selection, with the possibility of querying information for a certain area.
				\end{itemize}
				
				\item \textit{Abiding Geocast:}\cite{4531929}, which aims to deliver an Alert Message to a specific area where the warning is still relevant. Only vehicles that are travelling towards the effective area can participate in contention to broadcast the message. Moreover, broadcast is dynamically adjusted based on transmission range, speed, and distance between the potential forwarder and the destination area, increasing when such distance increases or the potential forwarder's speed decreases.
				
				\item \textit{Segment-oriented Data Abstraction and Dissemination
					(SODAD)}\cite{1402433}, a protocol according to which roads are divided into segments and each vehicle can both discover information itself and collect it from neighbor's transmissions. Whenever a vehicle receives a transmission from another vehicle, the information received will be classified as either one of two events:
				\begin{itemize}
					\item a \textit{provocation} event that will decrease the Broadcast Interval;
					\item a \textit{mollification} event that will increase the Broadcast Interval.
				\end{itemize}
				The classification is done via comparison of the newly received data with the information stored in the vehicle's on-board database. The vehicle assigns a higher weight if the difference between information coming from these two sources is high. The weight will be then compared against a threshold to establish whether a provocation of mollification event has taken place.
			\end{itemize}
		
		\subsection{Multi-hop Broadcasting Protocols}
		
		Multi-hop Broadcasting Protocols can be further subdivided into two categories:
		\begin{enumerate}
			\item Delay based protocols, which assign a different waiting time before rebroadcasting the message to each vehicle. This delay is usually inversely proportional to the distance between the source and the potential sender.
			Some examples are:
			\begin{itemize}
				\renewcommand\labelitemi{--}
				\item \textit{Urban Multi-hop Broadcast (UMB)} \cite{Korkmaz:2004:UMB:1023875.1023887}, designed to solve the broadcast redundancy, hidden node and reliability problems in multi-hop broadcasting using \textit{Request-to-Broadcast (RTB)} and \textit{Clear-To-Broadcast (CTB)} packets; 
				\item \textit{Smart Broadcast (SB)} \cite{4025102} and \textit{Efficient Directional Broadcast (EDB)} \cite{4340158}, which try to reduce the delay introduced by \textit{UMB} and remove the \textit{RTB} and \textit{CTB} packets, respectively;
				\item \textit{Vehicle-density-based Emergency Broadcasting (VDEB)} \cite{5663803}, a slotted broadcasting protocol which keeps vehicle density into consideration when computing waiting time slots;
				\item \textit{Reliable Method for Disseminating Safety Information
					(RMDSI)} \cite{4591259}, which aims to offer better performances when the network becomes fragmented by making a forwarder keep a copy of the packet it has broadcasted until it hears a retransmission (or until the packet lifetime expires). If no retransmission is heard within a certain time limit, the forwarder tries to find the next node which can relay the message using a small control packet;
				\item \textit{Multi-hop Vehicular Broadcast (MHVB)} \cite{4068699}, a protocol that keeps traffic congestion into consideration by   checking whether the number of neighbors of a vehicle is greather than a certain threshold and its speed is smaller than another threshold. When a node detects congestion, it increases its broadcast interval in order to try to reduce the network load;
				\item \textit{Reliable Broadcasting of Life Safety Messages (RBLSM)} \cite{4458046}, whose main objective is reliability, and a higher priority is given to the vehicle nearest to the sender instead to the one farthest from it, due to the assumption that the closer the vehicle is, the more reliable it is considered since its received signal strength is higher.
			\end{itemize}
			
			\item Probabilistic-based Multi-hop Broadcasting Protocols.
			The idea behind these kind of protocols is similar to the one behind Delay based protocols, but instead of assigning a different rebroadcast delay to each vehicle, a different rebroadcast probability is assigned. Each protocol differs in the function that assigns probabilities. Some examples of probabilistic-based protocols are:
			\begin{itemize}
				\renewcommand\labelitemi{--}
				\item \textit{Weighted p-Persistence} \cite{4407231}, in which every PFC computes its own rebroadcast probability based on distance between itself and the transmitter. The formula used is the following:
				\begin{gather}
					p_{ij} = \frac{D_{ij}}{R}
					\label{eq:weighted-p-persistence}
				\end{gather}
				where $D_{ij}$ is the distance between transmitter \textit{i} and PFC \textit{j} and R is the transmission range. Based on this function, the probability to rebroadcast is proportional to the distance between the PFC and the transmitter. The abovementioned formula does not keep into account vehicle density and also assumes that the transmission range is fixed and known to all vehicles.
				
				\item \textit{Optimized Adaptive Probabilistic Broadcast (OAPB)\cite{1543865} and AutoCast (AC) \cite{4350058}}, which both keep the vehicle density into consideration when computing the forwarding probability by making vehicle periodically exchange Hello Messages. Thanks to those messages, each vehicle can compute the number of neighbors and then use this information accordingly.
				
				\item \textit{Irresponsible Forwarding (IF)} \cite{4740277}\cite{5426212}, a protocol that considers vehicle density like \textit{OAPB} and \textit{AC}, but the formula used is not a simple linear function. In fact, the rebroadcast probability assignment function is the following:
				\begin{gather}
					p = e^{-\frac{\rho_s(z-d)}{c}}
				\end{gather}
				where $\rho_s$ is the vehicle density, $z$ is the transmission range, $d$ is the distance between the PFC and the transmitter and $c\geq1$ is a shaping parameter which influences the rebroadcast probability. \textit{Irresponsible Forwarding} aims to offer a solution that can scale with network density.
			\end{itemize}
		\end{enumerate}
		
			
			%		\item Network Coding-Based Multi-hop Broadcasting. TODO? da fare?
				The previous work\cite{ROM2017} focused on the implementation and testing of Fast-Broadcast, a Multi-hop delay based protocol firstly presented in \cite{4199282}. The main focus of the algorithm is to try to overcome the assumption of a fixed and known transmission range, which other protocols often tacitly assume. The protocol will be further analyzed in Chapter \ref{chapter:fb}.
				
				The authors of ROFF \cite{6906275}, another Multi-Hop delay based protocol, state that existing protocols are affected by two problems:
				\begin{itemize}
					\item the perfect suppression of redundant transmissions, by which potential forwarders which have lost the contention detect the transmission from the farthest vehicle and suppress their transmission. However this suppression can not always be guaranteed due to short difference between waiting times. In fact, if the timer of a potential forwarder expires before it has heard the transmission from the FFC, a redundant transmission will occur;
					\item the disuniformity and the costant change in spatial vehicle distribution in VANETs. Existing protocols which keep into consideration the distance between PFC and previous forwarder do not keep into consideration large empty spaces in the waiting time computation, leading to unnecessary wait.
				\end{itemize}
				ROFF's solutions to these problems and the implementation of the protocol will be analyzed in Chapter \ref{chapter:roff}.

	\section{Radio Propagation Models}
		Since field testing in VANETs, especially in large scenarios, is usually expensive and difficult to execute in real settings, researchers usually carry out their tests in a simulated environment, such as ns-3 (Section \ref{sec:ns3}). In order to model the transmission of signal throughout various media, several radio propagation models have been developed.
		A \gls{rpma} is an empirical mathematical formulation used to model the propagation of radio waves as a function of frequency, distance, transmission power and other variables. Over the years various RPMs have been developed, some aiming at modelling a general situation, and others more useful in specific scenarios. For example, implementations range from the more general free space model, where only distance and power are considered, to more complex models which account for shadowing, reflection, scattering, and other multipath losses. Moreover, it is important to keep into consideration the computational complexity and scalability of the model: some have poor accuracy but are scalable, while others have very good accuracy but can only work for small sets of nodes. As always, it is very important to find the right trade-off between complexity and accuracy.
		
		
		The authors of \cite{6298165} classify the propagation models offered by the network simulator ns-3 in three different categories:
		\begin{itemize}
			\item \textbf{Abstract} propagation loss models, for example the Maximal Range model (also known as Unit Disk), which establishes that all transmissions within a certain range are received without any loss;
			\item \textbf{Deterministic} path loss models, such as the Friis propagation model, which models quadratic path loss as it occurs in free space, and Two Ray Ground, which assume propagation via two rays: a direct (\acrshort{losa}) one, and the one reflected by the ground;
			\item \textbf{Stochastic} fading models such as the Nakagami model, which uses stochastic distributions to model path loss.
		\end{itemize}
	
	
		These traditional models, especially the stochastic ones, work quite well to describe the wireless channel characteristics from a macroscopic point of view. However, given the probabilistic nature of the model, single transmissions are not affected by the mesoscopic and microscopic effects of the sorrounding environment. To keep these effects into consideration, researchers have utilized Ray-Tracing, a geometrical optics technique used to determine all possible signal paths between the transmitter and the receiver, considering reflection, diffraction and scattering of radio waves, suitable both for 2D and 3D scenarios \cite{245274} \cite{765022}.
		
		
		However, a Ray-Tracing based approach, while producing a fairly accurate model, is not very scalable due to its high computational complexity, especially in a real-time scenario. To overcome this problem, the authors of \cite{STEPANOV200861} have resorted to a fairly computationally expensive pre-processing, but this leads to the need of pre-processing every scenario (and also every change in the scenario).
		
		
		The RPM utilized in this work will be presented in Section \ref{sec:shadowing}.
	
		
%		The previous thesis \cite{ROM2017} focused on the evaluation of Fast Broadcast \cite{4199282}, a Multi-Hop delay based protocol, through simulation in various scenarios
		
		\end{enumerate}  
% !TEX encoding = UTF-8
% !TEX TS-program = pdflatex
% !TEX root = ../tesi.tex

\chapter{Fast Broadcast}
	\label{chapter:fb}
	Fast Broadcast \cite{4199282} is a multi-hop routing protocol for vehicular communication. Its main feature consists in breaking the assumption that all vehicles should know, a priori, their fixed and constant transmission range. This assumption is often unreasonable, especially in \acrshort{vaneta}s and urban environments, where electromagnetic interferences and obstacles such as buildings heavily influence the transmission range.
	
	
	Fast Broadcast employs two different phases:
	\begin{enumerate}
		\item the \textbf{Estimation Phase}, during which cars estimate their frontward and backward transmission range;
		\item the \textbf{Broadcasts Phase}, during which a car sends an Alert Message and the other cars need to forward it in order to propagate the information.
	\end{enumerate}

	\section{Estimation Phase}
		During this phase, cars try to estimate their frontward and backward transmission range by the means of Hello Messages. These beacons are sent periodically via broadcast to all the neighbors of a vehicle.
		
		
		Time is divided into turns and, in order to keep estimations fresh, data collected during a certain turn is kept for the duration of the next turn, then discarded. The parameter \textit{turnSize} specifies the duration of a turn: the authors suggest a duration of one second. A bigger \textit{turnSize} could guarantee less collisions to the detriment of freshness of information. On the other hand, the effects of a smaller \textit{turnSize} are specular to those just presented. 
		
		
		Using this approach, vehicles can estimate two different values:
		\begin{itemize}
			\item \textit{Current-turn Maximum Front Range (CMFR)}, which estimates the maximum frontward distance from which another car can be heard by the considered one;
			\item \textit{Current-turn Maximum Back Range} (CMBR), which estimates the maximum backward distance at which the considered car can be heard.
		\end{itemize}
		When the turn expires, the value of these variables is stored in the LMFR and LMBR variables (\textit{Latest-turn Maximum Front Range} and \textit{Latest-turn Maximum Back Range}, respectively). The algorithm uses both last turn and current turn data because the former guarantees values calculated with a larger pool of Hello Messages, while the latter considers fresher information.
		
		When sending a Hello Message (Algorithm \ref{alg:hello-message-sending}), the vehicle initially waits for a random time between 0 and \textit{turnSize}. After this, if it has not heard another Hello Message or a collision, it proceeds to transmit a Hello Message containing the estimation of its frontward transmission range.
		
		
		When receiving a Hello Message (Algorithm \ref{alg:hello-message-receiving}), the vehicle retrieves its position and the sender's position, calculates the distance between these two positions and then updates the CMFR field if the message comes from ahead, otherwise CMBR is updated. The new value is obtained as the maximum between the old CMFR or CMBR value, the distance between the vehicle and the sender, and the sender's transmission range estimation included in the Hello Message.
		
		\begin{algorithm}[H]
			\begin{algorithmic}[1]
				\ForEach{turn}
					\State sendingTime $\gets$ random(turnSize)
					\State wait(sendingTime)
					\If{$\neg$ (heardHelloMsg() $\lor$ heardCollision())}
						\State helloMsg.declaredMaxRange $\gets$ max(LMFR, CMFR)
						\State transmit(helloMsg)
					\EndIf
				\EndFor
			\end{algorithmic}
			\caption{Hello message sending procedure}
			\label{alg:hello-message-sending}
		\end{algorithm}
		
		\begin{algorithm}[H]
			\begin{algorithmic}[1]
				\State mp $\gets$ myPosition()
				\State sp $\gets$ helloMsg.senderPosition
				\State drm $\gets$ helloMsg.declaredMaxRange
				\State d $\gets$ distance(mp, sp)
				\If{receivedFromFront(helloMsg)} 
				\State CMFR $\gets$ max(CMFR, d, drm)
				\Else
				\State CMBR $\gets$ max(CMBR, d, drm)
				\EndIf
			\end{algorithmic}
			\caption{Hello message receiving procedure}
			\label{alg:hello-message-receiving}
		\end{algorithm}
	
	\section{Broadcast Phase}
		This phase is activated once a car sends an Alert Message. The other cars can exploit the estimation of transmission ranges to reduce redundancy in message broadcast. Each vehicle can exploit this information to assign itself a forwarding priority inversely proportional to the relative distance: the higher the relative distance, the higher the priority.  
		
		
		When the Broadcast Phase is activated , a vehicle sends an Alert Message with application specific data. Broadcast specific data is also piggybacked on the Alert Message, such as:
		\begin{itemize}
			\item \textit{MaxRange:} the maximum range a transmission is expected to travel backward before the signal becomes too weak to be received. This value is utilized by following vehicles to rank their forwarding priority;
			\item \textit{SenderPosition}: the coordinates of the sender.
		\end{itemize}
		Upon reception, each vehicle waits for a random time called \textit{Contention Window} (\textit{CW}). This window ranges from a minimum value (\textit{CWMin}) and a maximum one (\textit{CWMax}) depending on sending/forwarding car distance (\textit{Distance}) and on the estimated transmission range (\textit{MaxRange}), according to formula \ref{eq:contention-window}. It is quite easy to see that the higher the sender/forwarder distance is, the lower the contention window is.
		\begin{gather}
			\left\lfloor \left( \frac{\text{MaxRange} - \text{Distance}}{\text{MaxRange}} \times (\text{CWMax} - \text{CWMin}) \right) + \text{CWmin}  \right\rfloor
			\label{eq:contention-window}
		\end{gather}
		If another forwarding of the same message coming from behind is heard during waiting time, the vehicle suppresses its transmission because the message has already been forwarded by another vehicle further back in the column. On the contrary, if the same message is heard coming from the front, the procedure is restarted using the new parameters. The vehicle can forward the message only if the waiting time expires without having received the same message.
		
		Algorithm \ref{alg:alert-message-generation} and \ref{alg:alert-message-forwarding} describe the logic behind the Broadcast Phase.
		
		\begin{algorithm}[H]
			\begin{algorithmic}[1]
				\State alertMessage.maxRange $\gets$ max(LMBR, CMBR)
				\State alertMessage.position $\gets$ retrievePosition()
				\State transmit(alertMessage) $\gets$ helloMsg.declaredMaxRange
			\end{algorithmic}
			\caption{Alert Message generation procedure}
			\label{alg:alert-message-generation}
		\end{algorithm}
	
		\begin{algorithm}[H]
			\begin{algorithmic}[1]
				\State cwnd $\gets$ computeCwnd()
				\State waitTime $\gets$ retrievePosition()
				\State wait(waitTime
				\If{sameBroadcastHeardFromBack()}
				\State exit()
				\ElsIf{sameBroadcastHeardFromFront()}
				\State restartBroadcastProcedure()
				\Else 
				\State maxRange $\gets$ max(LMBR, CMBR)
				\EndIf 
			\end{algorithmic}
			\caption{Alert Message generation procedure}
			\label{alg:alert-message-forwarding}
		\end{algorithm}
	
	\section{Two dimensions extension}
		The original work \cite{4199282} considered only a strip-shaped road, where it was easy to define directions and establish 	when a message came from the front or the back. In \cite{BAR2017} an extension considering two dimensions was proposed. 
		
		
		The modifications to the Fast Broadcast algorithm are the following:
		\begin{enumerate}
			\item Utilizing only one parameter between CMBR and CMFR (thus considering only CMR):
			\item Including the position of the vehicle which originally generated the Alert Message in addition to the position of the sender of the message.
		\end{enumerate}
		
		
		When a vehicle receives an Alert Message, the origin-vehicle distance is confronted with the origin-sender distance: the vehicle can forward the message only if the former is greater than or equal to the latter, otherwise it simply discards the message.
		
		\begin{figure}[H]
			\centering
			\includegraphics[width=\textwidth]{immagini/fb-2dpicc}
			\caption{Example of Fast Broadcast in 2d scenario}
			\label{fig:fb-2d}
		\end{figure}
		
		
		
		For example, suppose that vehicle A is the origin of the Alert Message and B receives it, but C doesn't due to an obstacle in the line of sight. B computes origin-vehicle distance, $d(A, B)$, and origin-sender distance, $d(A, A)$, which in this case are respectively 120 and 0m. Since origin-vehicle is greater than origin-sender, B can forward the Alert Message.
		
		
		Now suppose that C receives the message from B. C computes origin-vehicle distance, $d(A, C)$, and origin-sender distance, $d(A, B)$, which amount to 70 and 120m respectively. Since the former is not greater than or equal to the latter, C is not a candidate for forwarding and suppresses the transmission.
		
		
		D receives the message from B as well. D is a good candidate for transmission since the origin-vehicle distance, which amounts to 180m, is greater than origin-sender distance, equal to 120m.
%todo inserire struttura pacchetto
% todo rimuovere riferimenti alle distanze           
\input{capitoli/capitolo-3}            
%% !TEX encoding = UTF-8
% !TEX TS-program = pdflatex
% !TEX root = ../tesi.tex

\chapter{Analisi retrospettiva}
	
\section{Raggiungimento degli obiettivi}
	Nel complesso, lo \textit{stage} ha avuto un esito positivo. Nelle sezioni seguenti presento un resoconto degli obiettivi prefissati e raggiunti, utilizzando la seguente notazione
	\begin{itemize}
		\item \textbf{\greencheck:} obiettivo pienamente soddisfatto.
		\item \textbf{\yellowcheck:} obiettivo parzialmente soddisfatto.
		\item \textbf{\redx:} obiettivo non soddisfatto.
		
	\subsection{Progettuali}
		Includo una tabella riassuntiva che descrive il grado di raggiungimento degli obiettivi di progetto definiti nella sezione \ref{sec:obiettivi}.
		\end{itemize}
		
		\begin{tabularx}{\textwidth}{| X |c|}
			\hline
			\centering \textbf{Obiettivo} & \textbf{Soddisfacimento} \\
%				\Xhline{2\arrayrulewidth}
			\Xhline{2\arrayrulewidth}
			\multicolumn{2}{|l|}{\textbf{Obiettivi obbligatori}}\\
			\Xhline{2\arrayrulewidth}
			Studio e documentazione sulle differenze tra \textit{database} relazionale e \textit{Content Repository} & \greencheck\\
			\hline
			Studio e documentazione sulla storia di \textit{Content Repository} & \greencheck\\
			\hline
			Studio di JSR 170 e JSR283: Content Repository for Java (JCR), con produzione di codice e documentazione & \greencheck\\
			\hline
			Studio e documentazione della struttura di JCR & \greencheck\\
			\hline
			Studio e documentazione della definizione di nodo & \greencheck\\
			\hline
			Studio e documentazione riguardo aggiunta, rimozione e modifica di proprietà di un nodo & \greencheck\\
			\hline
			Studio e documentazione riguardo l'aggiunta e la rimozione di tipologie di nodo & \greencheck\\
			\hline
			Studio e documentazione riguardo la referenziazione di elementi & \greencheck\\
			\hline
			Studio e documentazione riguardo l'esecuzione di \textit{query} utilizzando XPath e JCR-SQL2 & \greencheck\\
			\hline
			Studio e documentazione riguardo l'indicizzazione & \greencheck\\
			\hline
			Progettazione di un prototipo di applicazione che gestisca le informazioni di prodotti commerciali & \greencheck\\
			\hline
			Realizzazione di un prototipo di applicazione che gestisca le informazioni di prodotti commerciali& \greencheck\\
			\Xhline{2\arrayrulewidth}
				\multicolumn{2}{|l|}{\textbf{Obiettivi desiderabili}}\\
			\Xhline{2\arrayrulewidth}
			Realizzazione della \gls{gui} del prototipo & \greencheck\\
			\Xhline{2\arrayrulewidth}
				\multicolumn{2}{|l|}{\textbf{Obiettivi facoltativi}}\\
			\Xhline{2\arrayrulewidth}
			Studio e documentazione riguardo i \textit{workspace} multipli & \yellowcheck\\
			\hline
			\caption{Resoconto soddisfacimento obiettivi del progetto.}
		\end{tabularx}

		Com'è possibile evincere dalla tabella ho completato tutti gli obiettivi obbligatori e desiderabili previsti dal piano di lavoro. Ho completato parzialmente l'obiettivo facoltativo riguardante i \textit{workspace} multipli in quanto ne ho effettuato uno studio solamente superficiale e non ho incluso tale funzionalità nel prototipo.
		
		A seguire includo una tabella di soddisfacimento delle funzionalità descritte nella sezione \ref{sec:requisiti}.
		
%			\def\arraystretch{1.5}
%			\rowcolors{2}{D}{P}
			\begin{tabularx}{\textwidth}{|l| X |c|}
%				\rowcolor{I}
%				\color{white} \textbf{Requisito} & \color{white} \textbf{Tipologia} & \color{white} \textbf{Descrizione} \\
%				\rowcolor{I} 
				\hline
				\textbf{Funzionalità} & \textbf{Importanza} & \textbf{Soddisfacimento} \\ 
				\hline
				Gestione prodotti & Obbligatoria & \greencheck\\
				\hline
				Gestione immagine prodotto & Desiderabile & \greencheck\\
				\hline
				Gestione categorie prodotti & Facoltativa & \yellowcheck\\
				\hline
				Gestione catalogo immagini prodotti & Facoltativa & \redx\\
				\hline
				\caption{Soddisfacimento requisiti.}
			\end{tabularx}
		
		Non ho implementato i requisiti opzionali che riguardavano le operazioni sulle tipologie di prodotto tramite interfaccia grafica per motivi di tempo. Tuttavia, ho fornito in ogni caso la possibilità di eseguire tali operazioni attraverso un file di configurazione scritto in linguaggio CND. In accordo con il \textit{tutor}, abbiamo ritenuto questa scelta accettabile dati i vincoli temporali, seppur non ottimale.
		
		
		
	
	\subsection{Aziendali}
		Procedo ad effettuare un'analisi sul raggiungimento degli obiettivi aziendali definiti nella sezione \ref{sec:motivazioni_aziendali}. Il grado di raggiungimento di tali obiettivi è solamente descritto dal mio punto di vista e in base alla mia percezione dello \textit{stage}. Di seguito non saranno quindi presenti opinioni del \textit{tutor} aziendale o di IBC.
		\begin{itemize}
			\item \textbf{Studio di nuove tecnologie} \greencheck\\
				L'azienda ha raggiunto pienamente questo obiettivo in quanto ho fornito documentazione, esempi di codice sorgente e un prototipo funzionante che illustrano il funzionamento di JCR e l'interazione con il \gls{framework} per la creazione di \gls{webapp} Wicket.
			\item \textbf{Assunzione di personale} \yellowcheck\\
				L'azienda ha raggiunto parzialmente questo obiettivo in quanto è riuscita a farmi apprendere in buona misura i meccanismi aziendali, il modo di lavorare e l'interfacciamento tra reparti. Tuttavia, data la mia intenzione di intraprendere l'istruzione universitaria magistrale, l'azienda non può vedere completato il suo obiettivo di assunzione immediata a fine \textit{stage}. In ogni caso, l'esperienza positiva di \textit{stage} e i risultati ottenuti non pregiudicano un'eventuale assunzione alla fine dei miei studi. 
			\item \textbf{Soluzioni originali} \redx \\
				L'azienda non ha raggiunto questo obiettivo in quanto le soluzioni da me implementate non sono né originali né innovative. Durante lo svolgimento del progetto non mi sono distanziato troppo da quanto dettato dagli \textit{standard} e da quanto suggerito dalle \textit{best pratice} aziendali e di settore. Personalmente reputo che prima di implementare una soluzione innovativa sia necessario conoscere profondamente le soluzioni classiche, obiettivo non raggiungibile secondo le mie capacità in un periodo di tempo di soli due mesi.
		\end{itemize} 
	
	\subsection{Personali}
		Complessivamente, considero raggiunti gli obiettivi personali definiti nella sezione \ref{sec:motivazioni_personali}. Segue un'analisi più approfondita.

		\begin{itemize}
			\item \textbf{Economici e logistici} \greencheck \\
				Ho raggiunto gli obiettivi economici e logistici in quanto l'azienda ha mantenuto la promessa di rimborso spese e la vicinanza alla sede universitaria mi ha permesso di completare con successo il progetto di \gls{swe}. 
			\item \textbf{Professionali} \yellowcheck \\
				Ho raggiunto solamente in parte gli obiettivi professionali. Ho potuto collaborare con un'azienda che combina consulenza a produzione di \textit{software} proprio, permettendomi di apprendere informazioni sul suo modo di lavorare. Tuttavia, non ho appreso tutte le conoscenze in ambito \gls{javaee} che speravo di apprendere, in quanto tale specifica è molto ampia e difficilmente applicabile in un progetto di breve durata.
			\item \textbf{Personali} \greencheck \\
				Ho raggiunto pienamente gli obiettivi personali in quanto mi sono rapportato con personale esperto e ho potuto avere consigli in tale ambito anche dopo la fine dello \textit{stage}.
		\end{itemize}
	
\section{Bilancio formativo personale}
	Nel complesso, il bilancio formativo personale dello \textit{stage} è positivo. Nelle sezioni seguenti fornisco una descrizione delle conoscenze, abilità e competenze che ho acquisito.
	\subsection{Conoscenze}
		Dagli studi effettuati durante lo svolgimento del progetto ho acquisito conoscenze nelle seguenti tecnologie da me non conosciute e nei seguenti campi di interesse:
		\begin{itemize}
			\item Le necessità e le problematiche riguardanti la persistenza dei dati.
			\item JCR e la libreria Jackrabbit.
			\item Apache Wicket.
			\item SVN, anche se solamente nel caso d'uso più semplice, con solo uno sviluppatore che contribuisce al \textit{repository}.
			\item Maven.
			\item Tomcat.
			\item JUnit.
			\item Eclipse.
			\item Superficialmente, la specifica \gls{javaee}.
		\end{itemize}
		
		Inoltre, per quanto riguarda le tecnologie già conosciute, ho approfondito l'utilizzo di:
		\begin{itemize}
			\item HTML5 e CSS.
			\item LibreOffice, sopratutto Writer e Calc.
			\item Sistema operativo Linux Mint.
		\end{itemize}
	
	\subsection{Abilità}
		Ho acquisito e approfondito varie abilità, tra cui:
			\begin{itemize}
				\item \textbf{Creazione di configurazioni \textit{software}} utilizzando Maven, in modo da gestire le dipendenze e il processo di \textit{build} di un progetto.
				\item \textbf{Implementazione di \textit{test}} per verificare la corretta risposta di un sistema a dei casi d'uso utilizzando JUnit. Considero questa abilità molto importante per poter effettuare la validazione con il committente in sicurezza.
				\item \textbf{\textit{Debug}} del codice a \textit{runtime} sfruttando le funzionalità offerte dall'IDE Eclipse.
				\item \textbf{Interazione con un'azienda} che sfrutta la metodologia Agile.
			\end{itemize}
	
	\subsection{Competenze}
		Infine, ho anche acquisito molte competenze. A seguire ne presento un elenco.
		\begin{itemize}
			\item \textbf{Progettazione, realizzazione e \textit{test}} di \textit{web app} basate sul \textit{framework} Wicket.
			\item \textbf{Conduzione di interviste} con un soggetto esterno (nel mio caso, il \textit{tutor} aziendale) per comprendere i requisiti di un prodotto \textit{software}. Nonostante avessi già una base di questa competenza grazie al progetto di \gls{swe}, l'interazione con il \textit{tutor} aziendale è stata la prima esperienza in cui ho condotto tali interviste completamente da solo. Questo mi ha quindi permesso di affinare la competenza.
			\item \textbf{Capacità nell'inserirmi in un contesto lavorativo} nuovo e di rapportarmi con colleghi e superiori, rispettando regole aziendali e modello di sviluppo. L'inserimento in un'azienda strutturata mi ha permesso lo sviluppo di questa competenza in misura maggiore rispetto ad un altro tipo di azienda.
			\item \textbf{Impiego della prototipazione} per fissare un'architettura \textit{software}, in modo da evitare costose riprogettazioni e cambiamenti architetturali.
			\item \textbf{Implementazione e comprensione di \textit{test} prestazionali}. Capire il modo con cui sono implementati i \textit{test} di questo tipo è importante per comprendere con efficacia i confronti tra diverse librerie e soluzioni proposte \textit{online}. 
		\end{itemize}	
	
\section{Mancanze nell'insegnamento accademico}
	Complessivamente, uno \textit{stage} dovrebbe fornire allo stagista un bagaglio di conoscenze nuove che normalmente non possono essere insegnate in ambito scolastico. Tuttavia, l'università dovrebbe dare le nozioni di base per poter svolgere con profitto lo \textit{stage}, senza la necessità di apprendere in ambiente lavorativo determinati argomenti. A seguire includo una lista delle conoscenze che, secondo la mia esperienza, dovrebbero essere inserite nell'insegnamento accademico.
	\begin{itemize}
		\item \textbf{Utilizzo di IDE e sistemi di versionamento del codice}. L'attuale assetto didattico non prevede alcuna introduzione all'utilizzo di IDE, uno strumento fondamentale per un programmatore. La prima occasione dove lo studente si trova ad utilizzare seriamente un IDE è durante il progetto di Ingegneria del \textit{software}, collocato alla fine del percorso di studi. Se il gruppo non approfondisce l'utilizzo dell'IDE, lo studente rischia di trovarsi spiazzato durante i primi giorni dello \textit{stage}. Stesso discorso vale per il versionamento del codice, fondamentale quando si lavora in gruppo.
			\begin{itemize}[leftmargin=*, labelsep=0pt, leftmargin=0pt]
				\item[] \textbf{Soluzione proposta:} attualmente le lezioni di laboratorio, specialmente nei primi due anni, sono poco guidate. Sfruttare alcune di queste lezioni, soprattutto durante i corsi di Programmazione e di Programmazione ad oggetti, per insegnare allo studente un utilizzo maturo di IDE e versionamento del codice potrebbe essere una buona soluzione.
			\end{itemize}
		\item \textbf{Inadeguatezza nell'insegnamento di tecnologie \textit{web}}. Lo studente che inizia lo \textit{stage} ha una conoscenza troppo basilare dell'ambito \textit{web}, in quanto l'omonimo corso del terzo anno tratta moltissime tecnologie in modo molto superficiale. Inoltre, il \textit{focus} sull'accessibilità concorre a ridurre ulteriormente il tempo a disposizione di altri argomenti. Alcune delle nozioni fondamentali che secondo me dovrebbero essere fornite sono il funzionamento di AJAX e dei servizi REST.
		\begin{itemize}[leftmargin=*, labelsep=0pt, leftmargin=0pt]
			\item[] \textbf{Soluzione proposta:}  una soluzione accettabile potrebbe essere trasferire parte delle tecnologie insegnate dal corso di Tecnologie \textit{web} al corso di Basi di dati, in modo da poter approfondire i temi che ho appena citato. Personalmente ripristinerei l'assetto didattico in vigore fino a tre anni fa, che prevedeva l'inclusione di argomenti come HTML e PHP nel corso di Basi. Tuttavia, per renderla una soluzione davvero efficace, sarebbe necessario eliminare la ridondanza presente a quei tempi, durante i quali il corso di Tecnologie \textit{web} riprendeva gli stessi argomenti (in ambito \textit{web}) trattati nel corso di Basi.
		\end{itemize}
		\item \textbf{Educazione alle tecnologie}. Il corso di studi vede gli studenti impegnati nei primi due anni a studiare tecnologie classiche e ormai ben affermate. Durante il progetto di Ingegneria del \textit{software} (e in alcuni \textit{stage}) viene richiesta la scelta dello \textit{stack} tecnologico, e nella maggior parte dei casi le alternative comprendono tecnologie innovative. Ritengo che gli strumenti a disposizione dello studente siano del tutto insufficienti per effettuare una scelta matura e consapevole. Se lo \textit{stack} tecnologico non è fissato, lo studente si trova a dover effettuare una scelta che, molto spesso, viene fatta in completa casualità o seguendo i consigli, a volte non oggettivi, presenti \textit{online}. Dato che non è possibile illustrare tutte le tecnologie esistenti, sarebbe utile dare come base un insieme di criteri che permettano allo studente di prendere decisioni più consapevoli in ambito tecnologico.
		\begin{itemize}[leftmargin=*, labelsep=0pt, leftmargin=0pt]
			\item[] \textbf{Soluzione proposta:} i seminari tenuti durante il corso di Ingegneria del \textit{software} sono un buon punto di partenza, ma il \textit{focus} di ogni seminario è una singola tecnologia, difficile da collocare nello \textit{stack} tecnologico per studenti inesperti. Inoltre, essi vengono tenuti durante uno dei periodi più impegnativi del percorso accademico, quindi la partecipazione non è sempre garantita.
			
			Una buona soluzione potrebbe essere tenere dei seminari nel corso del secondo anno che diano criteri di base sulla scelta dello \textit{stack} tecnologico e informazioni sullo stato dell'arte raggiunto in vari campi dell'informatica.
		\end{itemize}
		
	\end{itemize}
	
	           
%\input{capitoli/capitolo-4-old}             % Concept Preview
%\input{capitoli/capitolo-5-old}             % Product Prototype
%\input{capitoli/capitolo-6-old}             % Product Design Freeze e SOP
%\input{capitoli/capitolo-7-old}             % Conclusioni
%\appendix                               
%\input{capitoli/capitolo-A-old}             % Appendice A

%**************************************************************
% Materiale finale
%**************************************************************
\backmatter
\printglossaries
% !TEX encoding = UTF-8
% !TEX TS-program = pdflatex
% !TEX root = ../tesi.tex

%**************************************************************
% Bibliografia
%**************************************************************

\cleardoublepage
%\chapter{Bibliography}

\nocite{*}
% Stampa i riferimenti bibliografici
%\printbibliography[heading=subbibliography,title={Riferimenti bibliografici},type=book]

%\printbibliography[heading=subbibliography,title={Articles},type=article]

% Stampa i siti web consultati
%\bibliography{bibliografia}
\printbibliography
%\bibliographystyle{ieeetr}
\end{document}