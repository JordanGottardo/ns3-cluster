% !TEX encoding = UTF-8
% !TEX TS-program = pdflatex
% !TEX root = ../tesi.tex

%**************************************************************
% Sommario
%**************************************************************
\cleardoublepage
\phantomsection
\pdfbookmark{Abstract}{Abstract}
\begingroup
\let\clearpage\relax
\let\cleardoublepage\relax
\let\cleardoublepage\relax

\chapter*{Abstract}

The increasingly pervasive use of technology in the automotive industry and urban environment requires the development of broadcasting algorithms to deliver messages across vehicular ad-hoc networks (VANETs). These kinds of networks are created spontaneously and rely on vehicle-to-vehicle communication, without the need of any infrastructure or prior network topology knowledge by nodes. It is foreseeable that, in the near future, VANETs could be exploited in order to run heterogeneous applications, ranging from leisure-oriented functionalities such as video streaming and gaming, to more serious data exchanging services to monitor traffic congestion. One important application consists in emergency message distribution, where message delivery, timeliness and other life-safety related metrics are paramount. Whereas most of related work in this field is focused on one-dimensional topologies (i.e., car platooning in highways), this thesis consists in the reimplementation and redesign for two and three-dimensional scenarios of the RObust and Fast Forwarding algorithm (ROFF) and the Fast-Broadcast algorithm to compare them in various urban scenarios of increasing complexity. Considering urban scenarios, the Obstacle Model will be employed to take into account the shadowing effects of buildings on signal propagation. Moreover, this thesis proposes a Smart Junction extension for both algorithms, SJ-Fast-Broadcast and SJ-ROFF, which increase message delivery ratios by exploiting the presence of vehicles within road junctions in scenarios where the shadowing effects of obstacles is significant.

\endgroup			

\vfill

