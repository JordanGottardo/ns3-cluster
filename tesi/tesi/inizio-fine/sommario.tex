% !TEX encoding = UTF-8
% !TEX TS-program = pdflatex
% !TEX root = ../tesi.tex

%**************************************************************
% Sommario
%**************************************************************
\cleardoublepage
\phantomsection
\pdfbookmark{Abstract}{Abstract}
\begingroup
\let\clearpage\relax
\let\cleardoublepage\relax
\let\cleardoublepage\relax

\chapter*{Abstract}

The increasingly pervasive use of technology in the automotive industry and urban environment has lead to a need in the development of broadcasting algorithms to deliver messages across vehicular ad-hoc networks (VANETs). These kind of networks are created spontaneously and rely on vehicle-to-vehicle communication, without the need of any infrastructure or prior network topology knowledge by nodes. It is foreseeable that, in the near future, VANETs could be exploited in order to run applications of various kinds, ranging from more leisure-oriented functionalities such as video streaming and gaming, to more serious data exchanging services to monitor traffic congestion. One important application consists in emergency message distribution, where message delivery, timeliness and other life-safety related metrics are paramount. This thesis consists in the reimplementation and extension for two-dimensional scenarios of the RObust and Fast Forwarding algorithm (ROFF) and its comparison with the Fast-Broadcast scheme through simulations carried out in various scenarios of increasing complexity. For what concerns urban scenarios, the Obstacle Model will be employed to simulate the shadowing effects of buildings on signal propagation. Moreover, this thesis proposes a Smart Junction extension for both algorithms, SJ-Fast-Broadcast and SJ-ROFF, which increase message delivery ratios by exploiting vehicles inside road junctions in scenarios where the shadowing effects of obstacles is very severe.

\endgroup			

\vfill

