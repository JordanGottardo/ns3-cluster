% !TEX encoding = UTF-8
% !TEX TS-program = pdflatex
% !TEX root = ../tesi.tex

%**************************************************************
% Sommario
%**************************************************************
\cleardoublepage
\phantomsection
\pdfbookmark{Sommario}{Sommario}
\begingroup
\let\clearpage\relax
\let\cleardoublepage\relax
\let\cleardoublepage\relax

\chapter*{Sommario}

Il presente documento descrive il lavoro svolto durante il periodo di stage, della durata di trecentoventi ore, dal laureando Jordan Gottardo presso l'azienda IBC S.r.l. di Peraga (PD).


Gli obiettivi principali da raggiungere erano due. In primo luogo, veniva richiesto uno studio degli \textit{standard} \gls{jsr170} e \gls{jsr283}, che descrivono le API per l'utilizzo di Java Content Repository (JCR). Era richiesta la produzione di documentazione ed esempi di codice sorgente che sfruttassero tali API. 


Il secondo obiettivo riguardava l'implementazione di un prototipo, sottoforma di \gls{webapp}, che permettesse la memorizzazione di prodotti commerciali aventi attributi variabili. Era richiesta inoltre l'implementazione di funzionalità di ricerca per effettuare selezioni mirate di prodotti in più passi.


La libreria da utilizzare per la persistenza delle informazioni era Apache \gls{jackrabbit}, mentre il \gls{framework} per la realizzazione dell'interfaccia grafica era di libera scelta.


Il presente documento è organizzato in quattro capitoli:
\begin{itemize}
	\item \textbf{L'azienda:} in questo capitolo presento l'azienda che ha ospitato lo \textit{stage}, IBC S.r.l., fornendo descrizioni del contesto aziendale e del modo di lavorare. Descrivo inoltre i prodotti e i servizi che essa offre sul mercato.
	\item \textbf{L'offerta di \textit{stage}:} all'interno di questo capitolo descrivo il progetto di \textit{stage} offerto, soffermandomi sulle motivazioni aziendali e personali che hanno portato a questa scelta. Elencherò inoltre gli obiettivi da raggiungere.
	\item \textbf{Svolgimento del progetto:} in questo capitolo presento le attività svolte durante lo \textit{stage} per il raggiungimento degli obiettivi prefissati.
	\item \textbf{Analisi retrospettiva:}  all'interno di questo capitolo fornisco un'analisi retrospettiva sugli obiettivi dello \textit{stage}. Fornisco inoltre una descrizione di alcune mancanze nell'insegnamento accademico che dovrebbero essere aggiunte al piano didattico per un efficace approdo nel mondo del lavoro.
\end{itemize}

Nel documento utilizzerò le seguenti notazioni tipografiche:
\begin{itemize}
	\item \textit{Italico}: termine in lingua straniera.
	\item \texttt{Monospace}: nome di \textit{file}, classe o codice sorgente.
	\item Termine in azzurro: termine a glossario, solo la per la prima occorrenza di ogni capitolo. Cliccando sul termine è possibile leggere la spiegazione.
\end{itemize}

%\vfill
%
%\selectlanguage{english}
%\pdfbookmark{Abstract}{Abstract}
%\chapter*{Abstract}
%
%\selectlanguage{italian}

\endgroup			

\vfill

