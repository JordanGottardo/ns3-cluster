%**************************************************************
% file contenente le impostazioni della tesi
%**************************************************************

%**************************************************************
% Frontespizio
%**************************************************************

% Autore
\newcommand{\myName}{Jordan Gottardo}                                    
\newcommand{\myTitle}{Titolotitolotitolotitolo}

% Tipo di tesi                   
\newcommand{\myDegree}{Tesi di laurea magistrale}

% Università             
\newcommand{\myUni}{Università degli Studi di Padova}

% Facoltà       
\newcommand{\myFaculty}{Corso di Laurea in Informatica}

% Dipartimento
\newcommand{\myDepartment}{Dipartimento di Matematica "Tullio Levi-Civita"}

% Titolo del relatore
\newcommand{\profTitle}{Prof. }

% Relatore
\newcommand{\myProf}{Claudio E. Palazzi Armir Bujari}

% Luogo
\newcommand{\myLocation}{Padova}

% Anno accademico
\newcommand{\myAA}{2018-2019}

% Data discussione
\newcommand{\myTime}{????}


%**************************************************************
% Impostazioni di impaginazione
% see: http://wwwcdf.pd.infn.it/AppuntiLinux/a2547.htm
%**************************************************************

\setlength{\parindent}{14pt}   % larghezza rientro della prima riga
\setlength{\parskip}{0pt}   % distanza tra i paragrafi


%**************************************************************
% Impostazioni di biblatex
%**************************************************************
\bibliography{bibliografia} % database di biblatex 

\defbibheading{bibliography} {
    \cleardoublepage
    \phantomsection 
    \addcontentsline{toc}{chapter}{\bibname}
    \chapter*{\bibname\markboth{\bibname}{\bibname}}
}

\setlength\bibitemsep{1.5\itemsep} % spazio tra entry

\DeclareBibliographyCategory{opere}
\DeclareBibliographyCategory{web}

\addtocategory{opere}{womak:lean-thinking}
\addtocategory{web}{site:agile-manifesto}

\defbibheading{opere}{\section*{Riferimenti bibliografici}}
\defbibheading{web}{\section*{Siti Web consultati}}


%**************************************************************
% Impostazioni di caption
%**************************************************************
\captionsetup{
    tableposition=top,
    figureposition=bottom,
    font=small,
    format=hang,
    labelfont=bf
}

%**************************************************************
% Impostazioni di glossaries
%**************************************************************

%**************************************************************
% Acronimi
%**************************************************************
\renewcommand{\acronymname}{Acronimi e abbreviazioni}

%\newacronym[description={\glslink{rpma}{rpm}}]{rpma}{RPM}{Radio Propagation Model}
\newacronym{rpma}{RPM}{Radio Propagation Model}
\newacronym{vaneta}{VANET}{Vehicular Ad-Hoc Network}
\newacronym{losa}{LOS}{Line of sight}
\newacronym{nica}{NIC}{Network Interface Controller}
%\newacronym[description={Unified Modeling Language}}]
%{uml}{UML}{Unified Modeling Language}

%**************************************************************
% Glossario
%**************************************************************
%\renewcommand{\glossaryname}{Glossario}

%**************************************************************
% Termini Glossario Jordan
%**************************************************************

%\newglossaryentry{rpmg} {
%	name=\glslink{rpm}{RPM},
%	text=\mbox{JSR 170},
%	sort=rpm,
%	description={Descrizione rpm}
%}


%\newglossaryentry{jsr283} {
%	name=\glslink{jsr283}{JSR 283},
%	text=\mbox{JSR 283},
%	sort=jsr283,
%	description={Java Request Specification rilasciato il 25 settembre 2009. Rispetto a JSR 170, aggiunge (e in alcuni casi rimpiazza) alcune API e funzionalità. È conosciuto anche come \jquote{JCR v2.0 Specifications}}
%}
%
%\newglossaryentry{webapp} {
%	name=\glslink{webapp}{\textit{Web app}},
%	text=\textit{web app},
%	sort=webapp,
%	description={(ing. applicazione web). È un'applicazione fruibile tramite \textit{web browser}}
%}
%
%\newglossaryentry{framework} {
%	name=\glslink{framework}{Framework},
%	text=\textit{framework},
%	sort=framework,
%	description={È un'architettura logica di supporto (spesso un'implementazione logica di un particolare \textit{design pattern}) su cui un \textit{software} può essere progettato e realizzato, spesso facilitandone lo sviluppo da parte del programmatore}
%}
%
%\newglossaryentry{fidelity} {
%	name=\glslink{fidelity}{Fidelity},
%	text=\textit{fidelity},
%	sort=fidelity,
%	description={È un insieme di pratiche attuate da un'organizzazione commerciale per favorire la fidelizzazione della clientela attraverso premi, agevolazioni e altri incentivi all’acquisto come la classica raccolta punti}
%}
%
%\newglossaryentry{NCR} {
%	name=\glslink{NCR}{NCR},
%	text=NCR,
%	sort=NCR,
%	description={Sigla di National Cash Register. È un'azienda fondata nel 1884 che attualmente opera in gran parte del mondo con soluzioni \textit{retail} e \textit{financial}. Ha sede principale a Dayton (Ohio), U.S.A.; la sede italiana è situata a Milano. Produce principalmente ATM e registratori di cassa}
%}
%
%\newglossaryentry{POS} {
%	name=\glslink{POS}{POS},
%	text=POS,
%	sort=POS,
%	description={(ing. POS, \textit{Point of Sale}). È il dispositivo elettronico che permette di effettuare pagamenti mediante moneta elettronica, ovvero tramite carte di credito, di debito e prepagate}
%}
%
%\newglossaryentry{retail} {
%	name=\glslink{retail}{\textit{Retail}},
%	text=\textit{retail},
%	sort=retail,
%	description={(ing. vendita al dettaglio). È una locuzione utilizzata in ambito commerciale per indicare la vendita di prodotti al consumatore finale. È l'ultimo anello della catena di distribuzione, che inizia dal produttore e può passare per un certo numero di grossisti}
%}
%
%\newglossaryentry{GDO} {
%	name=\glslink{GDO}{GDO},
%	text=GDO,
%	sort=GDO,
%	description={Sigla di Grande Distribuzione Organizzata. Si riferisce al moderno sistema di vendita al dettaglio attraverso una rete di supermercati e ipermercati e di altre catene di intermediari di varia natura. Rappresenta l'evoluzione del supermercato singolo, che a sua volta costituisce lo sviluppo del negozio tradizionale}
%}
%
%\newglossaryentry{clickandcollect} {
%	name=\glslink{clickandcollect}{Click \& collect},
%	text=\textit{click \& collect},
%	sort=click\&collect,
%	description={(ing. prenota e ritira). Metodo di vendita al dettaglio che consiste nella prenotazione, solitamente via \textit{web}, del prodotto da parte del cliente e nel successivo ritiro quando viene segnalata la disponibilità della merce ordinata. La differenza con l'\textit{e-commerce} classico è che la spedizione (in questo caso il ritiro) viene effettuata direttamente dal cliente, senza l'ausilio di corrieri}
%}
%
%\newglossaryentry{pda} {
%	name=\glslink{pda}{PDA},
%	text=PDA,
%	sort=PDA,
%	description={Sigla di \textit{Personal Digital Assistant}. Indica un computer palmare, ovvero un computer di dimensioni talmente contenute da poter essere portato sul palmo di una mano. Lo schermo del PDA è tattile, in modo da permettere l'interazione con le dita o con un apposito pennino}
%}
%
%\newglossaryentry{backoffice} {
%	name=\glslink{backoffice}{\textit{Back office}},
%	text=\textit{back office},
%	sort=backoffice,
%	description={(ing. dietro ufficio, nel significato di retro-ufficio). Termine che indica la parte di azienda che comprende le attività di gestione operativa, amministrativa e tutte le attività che non riguardano direttamente il cliente}
%}
%
%\newglossaryentry{opensource} {
%	name=\glslink{opensource}{\textit{Open source}},
%	text=\textit{open source},
%	sort=opensource,
%	description={(ing. sorgente aperta). È un termine che indica un \textit{software} di cui i detentori dei diritti rendono pubblico il codice sorgente. Così facendo, altri programmatori possono studiare il codice e apportarvi liberamente modifiche ed estensioni}
%}
%
%\newglossaryentry{webservice} {
%	name=\glslink{webservice}{\textit{Web service}},
%	text=\textit{web service},
%	sort=webservice,
%	description={Tipo di architettura \textit{software} che si basa sulla comunicazione tra sistemi distribuiti. La comunicazione solitamente avviene solitamente utilizzando linguaggi come XML e JSON, con messaggi trasportati da protocolli \textit{web} (da cui il nome), come HTTP}
%}
%
%\newglossaryentry{proofofconcept} {
%	name=\glslink{proofofconcept}{\textit{Proof of concept}},
%	text=\textit{proof of concept},
%	sort=proofofconcept,
%	description={(ing. prova del concetto). Termine che indica un prototipo o un'incompleta realizzazione di un progetto, in modo da poterne dimostrare la sua fattibilità}
%}
%
%\newglossaryentry{jackrabbit} {
%	name=\glslink{jackrabbit}{Jackrabbit},
%	text=Jackrabbit,
%	sort=jackrabbit,
%	description={Apache Jackrabbit è una libreria Java open source che fornisce un'implementazione di un Java Content Repository, così come definito dagli standard JSR 170 e JSR 283}
%}
%
%\newglossaryentry{gui} {
%	name=\glslink{gui}{\textit{GUI}},
%	text=GUI,
%	sort=gui,
%	description={(ing. \textit{Graphical User Interface}, interfaccia grafica utente). Indica l'interfaccia con cui l'utente interagisce con un \textit{software} attraverso il controllo di oggetti grafici convenzionali}
%}
%
%\newglossaryentry{javaee} {
%	name=\glslink{javaee}{Java EE},
%	text=Java EE,
%	sort=javaee,
%	description={Java Platform, Enterprise Edition. È una specifica impiegata nello sviluppo di applicazioni \textit{web} in linguaggio Java. Inizialmente, la specifica puntava verso la creazione di applicazioni con architetture \textit{multi-tier}, ma grazie alle recenti evoluzioni permette anche di creare applicazioni basate su microservizi}
%}
%
%\newglossaryentry{swe} {
%	name=\glslink{swe}{Ingegneria del \textit{software}},
%	text=Ingegneria del \textit{software},
%	sort=ingegneriadelsoftware,
%	description={Corso della Laurea Triennale in Informatica di Padova che richiede lo sviluppo di un impegnativo progetto didattico di gruppo secondo canoni rigorosi di gestione del rapporto cliente-fornitore}
%}
%
%\newglossaryentry{fulltext} {
%	name=\glslink{fulltext}{\textit{Full-text}},
%	text=\textit{full-text},
%	sort=fulltext,
%	description={(ing. testo intero). Indica un tipo di ricerca testuale all'interno di un documento o di un \textit{database} in cui il motore di ricerca esamina tutte le parole memorizzate e tenta di trovare un riscontro secondo determinate parole fornite dall'utente}
%}
%
%\newglossaryentry{stakeholder} {
%	name=\glslink{stakeholder}{\textit{Stakeholder}},
%	text=\textit{stakeholder},
%	sort=stakeholder,
%	description={(ing. portatore di interessi). In economia, indica un soggetto che esercita influenza nei confronti di un'attività economica, come ad esempio un progetto}
%}
%
%\newglossaryentry{stub} {
%	name=\glslink{stub}{\textit{Stub}},
%	text=\textit{stub},
%	sort=stub,
%	description={(ing. abbozzo). Indica una porzione di codice utilizzata in sostituzione di altre funzionalità \textit{software}. È utilizzato sopratutto durante l'esecuzione dei \textit{test} per simulare il comportamento di codice su cui non si sta eseguendo il \textit{test}}
%}
%
%\newglossaryentry{statementcoverage} {
%	name=\glslink{statementcoverage}{\textit{Statement coverage}},
%	text=\textit{Statement coverage},
%	sort=statementcoverage,
%	description={(ing. copertura delle dichiarazioni). Indica il grado di codice sorgente che viene eseguito durante l'attuazione dei \textit{test}}
%}
%
%\newglossaryentry{branchcoverage} {
%	name=\glslink{branchcoverage}{\textit{Branch coverage}},
%	text=\textit{Branch coverage},
%	sort=branchcoverage,
%	description={(ing. copertura dei rami). Indica il grado di cammini logici all'interno di un programma coperti durante l'esecuzione dei \textit{test}}
%}
%
%\newglossaryentry{incrementi} {
%	name=\glslink{incrementi}{Incremento},
%	text=\textit{incrementi},
%	sort=incrementi,
%	description={Nel modello incrementale, un incremento è un aumento tangibile di valore del \textit{software} in fase di sviluppo. Il caso più comune di incremento è l'introduzione di nuove funzionalità. Un incremento, per essere tale, deve essere validato internamente}
%}
%
%%**************************************************************
%% Termini Glossario esempio
%%**************************************************************
%
%%\newglossaryentry{apig}
%%{
%%    name=\glslink{api}{API},
%%    text=Application Program Interface,
%%    sort=api,
%%    description={in informatica con il termine \emph{Application Programming Interface API} (ing. interfaccia di programmazione di un'applicazione) si indica ogni insieme di procedure disponibili al programmatore, di solito raggruppate a formare un set di strumenti specifici per l'espletamento di un determinato compito all'interno di un certo programma. La finalità è ottenere un'astrazione, di solito tra l'hardware e il programmatore o tra \textit{software} a basso e quello ad alto livello semplificando così il lavoro di programmazione}
%%}
%
%
%%\newglossaryentry{umlg}
%%{
%%    name=\glslink{uml}{UML},
%%    text=UML,
%%    sort=uml,
%%    description={in ingegneria del \textit{software} \emph{UML, Unified Modeling Language} (ing. linguaggio di modellazione unificato) è un linguaggio di modellazione e specifica basato sul paradigma object-oriented. L'\emph{UML} svolge un'importantissima funzione di ``lingua franca'' nella comunità della progettazione e programmazione a oggetti. Gran parte della letteratura di settore usa tale linguaggio per descrivere soluzioni analitiche e progettuali in modo sintetico e comprensibile a un vasto pubblico}
%%}
 % database di termini
\makeglossaries


%**************************************************************
% Impostazioni di graphicx
%**************************************************************
\graphicspath{{immagini/}} % cartella dove sono riposte le immagini


%**************************************************************
% Impostazioni di hyperref
%**************************************************************
\hypersetup{
    %hyperfootnotes=false,
    %pdfpagelabels,
    %draft,	% = elimina tutti i link (utile per stampe in bianco e nero)
    colorlinks=true,
    linktocpage=true,
    pdfstartpage=1,
    pdfstartview=FitV,
    % decommenta la riga seguente per avere link in nero (per esempio per la stampa in bianco e nero)
    %colorlinks=false, linktocpage=false, pdfborder={0 0 0}, pdfstartpage=1, pdfstartview=FitV,
    breaklinks=true,
    pdfpagemode=UseNone,
    pageanchor=true,
    pdfpagemode=UseOutlines,
    plainpages=false,
    bookmarksnumbered,
    bookmarksopen=true,
    bookmarksopenlevel=1,
    hypertexnames=true,
    pdfhighlight=/O,
    %nesting=true,
    %frenchlinks,
    urlcolor=webbrown,
    linkcolor=RoyalBlue,
    citecolor=webgreen,
    %pagecolor=RoyalBlue,
    %urlcolor=Black, linkcolor=Black, citecolor=Black, %pagecolor=Black,
    pdftitle={\myTitle},
    pdfauthor={\textcopyright\ \myName, \myUni, \myFaculty},
    pdfsubject={},
    pdfkeywords={},
    pdfcreator={pdfLaTeX},
    pdfproducer={LaTeX}
}

%**************************************************************
% Impostazioni di itemize
%**************************************************************
\renewcommand{\labelitemi}{$\ast$}

%\renewcommand{\labelitemi}{$\bullet$}
%\renewcommand{\labelitemii}{$\cdot$}
%\renewcommand{\labelitemiii}{$\diamond$}
%\renewcommand{\labelitemiv}{$\ast$}


%**************************************************************
% Impostazioni di listings
%**************************************************************
\lstset{
    language=[LaTeX]Tex,%C++,
    keywordstyle=\color{RoyalBlue}, %\bfseries,
    basicstyle=\small\ttfamily,
    %identifierstyle=\color{NavyBlue},
    commentstyle=\color{Green}\ttfamily,
    stringstyle=\rmfamily,
    numbers=none, %left,%
    numberstyle=\scriptsize, %\tiny
    stepnumber=5,
    numbersep=8pt,
    showstringspaces=false,
    breaklines=true,
    frameround=ftff,
    frame=single
} 


%**************************************************************
% Impostazioni di xcolor
%**************************************************************
\definecolor{webgreen}{rgb}{0,.5,0}
\definecolor{webbrown}{rgb}{.6,0,0}


%**************************************************************
% Altro
%**************************************************************

\newcommand{\omissis}{[\dots\negthinspace]} % produce [...]

% eccezioni all'algoritmo di sillabazione
\hyphenation
{
    ma-cro-istru-zio-ne
    gi-ral-din
}

\newcommand{\sectionname}{sezione}
\addto\captionsitalian{\renewcommand{\figurename}{Figura}
                       \renewcommand{\tablename}{Tabella}}

\newcommand{\glsfirstoccur}{\ap{{[g]}}}

\newcommand{\intro}[1]{\emph{\textsf{#1}}}

%**************************************************************
% Environment per ``rischi''
%**************************************************************
\newcounter{riskcounter}                % define a counter
\setcounter{riskcounter}{0}             % set the counter to some initial value

%%%% Parameters
% #1: Title
\newenvironment{risk}[1]{
    \refstepcounter{riskcounter}        % increment counter
    \par \noindent                      % start new paragraph
    \textbf{\arabic{riskcounter}. #1}   % display the title before the 
                                        % content of the environment is displayed 
}{
    \par\medskip
}

\newcommand{\riskname}{Rischio}

\newcommand{\riskdescription}[1]{\textbf{\\Descrizione:} #1.}

\newcommand{\risksolution}[1]{\textbf{\\Soluzione:} #1.}

%**************************************************************
% Environment per ``use case''
%**************************************************************
\newcounter{usecasecounter}             % define a counter
\setcounter{usecasecounter}{0}          % set the counter to some initial value

%%%% Parameters
% #1: ID
% #2: Nome
\newenvironment{usecase}[2]{
    \renewcommand{\theusecasecounter}{\usecasename #1}  % this is where the display of 
                                                        % the counter is overwritten/modified
    \refstepcounter{usecasecounter}             % increment counter
    \vspace{10pt}
    \par \noindent                              % start new paragraph
    {\large \textbf{\usecasename #1: #2}}       % display the title before the 
                                                % content of the environment is displayed 
    \medskip
}{
    \medskip
}

\newcommand{\usecasename}{UC}

\newcommand{\usecaseactors}[1]{\textbf{\\Attori Principali:} #1. \vspace{4pt}}
\newcommand{\usecasepre}[1]{\textbf{\\Precondizioni:} #1. \vspace{4pt}}
\newcommand{\usecasedesc}[1]{\textbf{\\Descrizione:} #1. \vspace{4pt}}
\newcommand{\usecasepost}[1]{\textbf{\\Postcondizioni:} #1. \vspace{4pt}}
\newcommand{\usecasealt}[1]{\textbf{\\Scenario Alternativo:} #1. \vspace{4pt}}

%**************************************************************
% Environment per ``namespace description''
%**************************************************************

\newenvironment{namespacedesc}{
    \vspace{10pt}
    \par \noindent                              % start new paragraph
    \begin{description} 
}{
    \end{description}
    \medskip
}

\newcommand{\classdesc}[2]{\item[\textbf{#1:}] #2}

%**************************************************************
% Comandi Jordan
%**************************************************************

\newcommand{\jquote}[1]{“#1”}