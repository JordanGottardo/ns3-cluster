% !TEX encoding = UTF-8
% !TEX TS-program = pdflatex
% !TEX root = ../tesi.tex

\chapter{Analisi retrospettiva}
	
\section{Raggiungimento degli obiettivi}
	Nel complesso, lo \textit{stage} ha avuto un esito positivo. Nelle sezioni seguenti presento un resoconto degli obiettivi prefissati e raggiunti, utilizzando la seguente notazione
	\begin{itemize}
		\item \textbf{\greencheck:} obiettivo pienamente soddisfatto.
		\item \textbf{\yellowcheck:} obiettivo parzialmente soddisfatto.
		\item \textbf{\redx:} obiettivo non soddisfatto.
		
	\subsection{Progettuali}
		Includo una tabella riassuntiva che descrive il grado di raggiungimento degli obiettivi di progetto definiti nella sezione \ref{sec:obiettivi}.
		\end{itemize}
		
		\begin{tabularx}{\textwidth}{| X |c|}
			\hline
			\centering \textbf{Obiettivo} & \textbf{Soddisfacimento} \\
%				\Xhline{2\arrayrulewidth}
			\Xhline{2\arrayrulewidth}
			\multicolumn{2}{|l|}{\textbf{Obiettivi obbligatori}}\\
			\Xhline{2\arrayrulewidth}
			Studio e documentazione sulle differenze tra \textit{database} relazionale e \textit{Content Repository} & \greencheck\\
			\hline
			Studio e documentazione sulla storia di \textit{Content Repository} & \greencheck\\
			\hline
			Studio di JSR 170 e JSR283: Content Repository for Java (JCR), con produzione di codice e documentazione & \greencheck\\
			\hline
			Studio e documentazione della struttura di JCR & \greencheck\\
			\hline
			Studio e documentazione della definizione di nodo & \greencheck\\
			\hline
			Studio e documentazione riguardo aggiunta, rimozione e modifica di proprietà di un nodo & \greencheck\\
			\hline
			Studio e documentazione riguardo l'aggiunta e la rimozione di tipologie di nodo & \greencheck\\
			\hline
			Studio e documentazione riguardo la referenziazione di elementi & \greencheck\\
			\hline
			Studio e documentazione riguardo l'esecuzione di \textit{query} utilizzando XPath e JCR-SQL2 & \greencheck\\
			\hline
			Studio e documentazione riguardo l'indicizzazione & \greencheck\\
			\hline
			Progettazione di un prototipo di applicazione che gestisca le informazioni di prodotti commerciali & \greencheck\\
			\hline
			Realizzazione di un prototipo di applicazione che gestisca le informazioni di prodotti commerciali& \greencheck\\
			\Xhline{2\arrayrulewidth}
				\multicolumn{2}{|l|}{\textbf{Obiettivi desiderabili}}\\
			\Xhline{2\arrayrulewidth}
			Realizzazione della \gls{gui} del prototipo & \greencheck\\
			\Xhline{2\arrayrulewidth}
				\multicolumn{2}{|l|}{\textbf{Obiettivi facoltativi}}\\
			\Xhline{2\arrayrulewidth}
			Studio e documentazione riguardo i \textit{workspace} multipli & \yellowcheck\\
			\hline
			\caption{Resoconto soddisfacimento obiettivi del progetto.}
		\end{tabularx}

		Com'è possibile evincere dalla tabella ho completato tutti gli obiettivi obbligatori e desiderabili previsti dal piano di lavoro. Ho completato parzialmente l'obiettivo facoltativo riguardante i \textit{workspace} multipli in quanto ne ho effettuato uno studio solamente superficiale e non ho incluso tale funzionalità nel prototipo.
		
		A seguire includo una tabella di soddisfacimento delle funzionalità descritte nella sezione \ref{sec:requisiti}.
		
%			\def\arraystretch{1.5}
%			\rowcolors{2}{D}{P}
			\begin{tabularx}{\textwidth}{|l| X |c|}
%				\rowcolor{I}
%				\color{white} \textbf{Requisito} & \color{white} \textbf{Tipologia} & \color{white} \textbf{Descrizione} \\
%				\rowcolor{I} 
				\hline
				\textbf{Funzionalità} & \textbf{Importanza} & \textbf{Soddisfacimento} \\ 
				\hline
				Gestione prodotti & Obbligatoria & \greencheck\\
				\hline
				Gestione immagine prodotto & Desiderabile & \greencheck\\
				\hline
				Gestione categorie prodotti & Facoltativa & \yellowcheck\\
				\hline
				Gestione catalogo immagini prodotti & Facoltativa & \redx\\
				\hline
				\caption{Soddisfacimento requisiti.}
			\end{tabularx}
		
		Non ho implementato i requisiti opzionali che riguardavano le operazioni sulle tipologie di prodotto tramite interfaccia grafica per motivi di tempo. Tuttavia, ho fornito in ogni caso la possibilità di eseguire tali operazioni attraverso un file di configurazione scritto in linguaggio CND. In accordo con il \textit{tutor}, abbiamo ritenuto questa scelta accettabile dati i vincoli temporali, seppur non ottimale.
		
		
		
	
	\subsection{Aziendali}
		Procedo ad effettuare un'analisi sul raggiungimento degli obiettivi aziendali definiti nella sezione \ref{sec:motivazioni_aziendali}. Il grado di raggiungimento di tali obiettivi è solamente descritto dal mio punto di vista e in base alla mia percezione dello \textit{stage}. Di seguito non saranno quindi presenti opinioni del \textit{tutor} aziendale o di IBC.
		\begin{itemize}
			\item \textbf{Studio di nuove tecnologie} \greencheck\\
				L'azienda ha raggiunto pienamente questo obiettivo in quanto ho fornito documentazione, esempi di codice sorgente e un prototipo funzionante che illustrano il funzionamento di JCR e l'interazione con il \gls{framework} per la creazione di \gls{webapp} Wicket.
			\item \textbf{Assunzione di personale} \yellowcheck\\
				L'azienda ha raggiunto parzialmente questo obiettivo in quanto è riuscita a farmi apprendere in buona misura i meccanismi aziendali, il modo di lavorare e l'interfacciamento tra reparti. Tuttavia, data la mia intenzione di intraprendere l'istruzione universitaria magistrale, l'azienda non può vedere completato il suo obiettivo di assunzione immediata a fine \textit{stage}. In ogni caso, l'esperienza positiva di \textit{stage} e i risultati ottenuti non pregiudicano un'eventuale assunzione alla fine dei miei studi. 
			\item \textbf{Soluzioni originali} \redx \\
				L'azienda non ha raggiunto questo obiettivo in quanto le soluzioni da me implementate non sono né originali né innovative. Durante lo svolgimento del progetto non mi sono distanziato troppo da quanto dettato dagli \textit{standard} e da quanto suggerito dalle \textit{best pratice} aziendali e di settore. Personalmente reputo che prima di implementare una soluzione innovativa sia necessario conoscere profondamente le soluzioni classiche, obiettivo non raggiungibile secondo le mie capacità in un periodo di tempo di soli due mesi.
		\end{itemize} 
	
	\subsection{Personali}
		Complessivamente, considero raggiunti gli obiettivi personali definiti nella sezione \ref{sec:motivazioni_personali}. Segue un'analisi più approfondita.

		\begin{itemize}
			\item \textbf{Economici e logistici} \greencheck \\
				Ho raggiunto gli obiettivi economici e logistici in quanto l'azienda ha mantenuto la promessa di rimborso spese e la vicinanza alla sede universitaria mi ha permesso di completare con successo il progetto di \gls{swe}. 
			\item \textbf{Professionali} \yellowcheck \\
				Ho raggiunto solamente in parte gli obiettivi professionali. Ho potuto collaborare con un'azienda che combina consulenza a produzione di \textit{software} proprio, permettendomi di apprendere informazioni sul suo modo di lavorare. Tuttavia, non ho appreso tutte le conoscenze in ambito \gls{javaee} che speravo di apprendere, in quanto tale specifica è molto ampia e difficilmente applicabile in un progetto di breve durata.
			\item \textbf{Personali} \greencheck \\
				Ho raggiunto pienamente gli obiettivi personali in quanto mi sono rapportato con personale esperto e ho potuto avere consigli in tale ambito anche dopo la fine dello \textit{stage}.
		\end{itemize}
	
\section{Bilancio formativo personale}
	Nel complesso, il bilancio formativo personale dello \textit{stage} è positivo. Nelle sezioni seguenti fornisco una descrizione delle conoscenze, abilità e competenze che ho acquisito.
	\subsection{Conoscenze}
		Dagli studi effettuati durante lo svolgimento del progetto ho acquisito conoscenze nelle seguenti tecnologie da me non conosciute e nei seguenti campi di interesse:
		\begin{itemize}
			\item Le necessità e le problematiche riguardanti la persistenza dei dati.
			\item JCR e la libreria Jackrabbit.
			\item Apache Wicket.
			\item SVN, anche se solamente nel caso d'uso più semplice, con solo uno sviluppatore che contribuisce al \textit{repository}.
			\item Maven.
			\item Tomcat.
			\item JUnit.
			\item Eclipse.
			\item Superficialmente, la specifica \gls{javaee}.
		\end{itemize}
		
		Inoltre, per quanto riguarda le tecnologie già conosciute, ho approfondito l'utilizzo di:
		\begin{itemize}
			\item HTML5 e CSS.
			\item LibreOffice, sopratutto Writer e Calc.
			\item Sistema operativo Linux Mint.
		\end{itemize}
	
	\subsection{Abilità}
		Ho acquisito e approfondito varie abilità, tra cui:
			\begin{itemize}
				\item \textbf{Creazione di configurazioni \textit{software}} utilizzando Maven, in modo da gestire le dipendenze e il processo di \textit{build} di un progetto.
				\item \textbf{Implementazione di \textit{test}} per verificare la corretta risposta di un sistema a dei casi d'uso utilizzando JUnit. Considero questa abilità molto importante per poter effettuare la validazione con il committente in sicurezza.
				\item \textbf{\textit{Debug}} del codice a \textit{runtime} sfruttando le funzionalità offerte dall'IDE Eclipse.
				\item \textbf{Interazione con un'azienda} che sfrutta la metodologia Agile.
			\end{itemize}
	
	\subsection{Competenze}
		Infine, ho anche acquisito molte competenze. A seguire ne presento un elenco.
		\begin{itemize}
			\item \textbf{Progettazione, realizzazione e \textit{test}} di \textit{web app} basate sul \textit{framework} Wicket.
			\item \textbf{Conduzione di interviste} con un soggetto esterno (nel mio caso, il \textit{tutor} aziendale) per comprendere i requisiti di un prodotto \textit{software}. Nonostante avessi già una base di questa competenza grazie al progetto di \gls{swe}, l'interazione con il \textit{tutor} aziendale è stata la prima esperienza in cui ho condotto tali interviste completamente da solo. Questo mi ha quindi permesso di affinare la competenza.
			\item \textbf{Capacità nell'inserirmi in un contesto lavorativo} nuovo e di rapportarmi con colleghi e superiori, rispettando regole aziendali e modello di sviluppo. L'inserimento in un'azienda strutturata mi ha permesso lo sviluppo di questa competenza in misura maggiore rispetto ad un altro tipo di azienda.
			\item \textbf{Impiego della prototipazione} per fissare un'architettura \textit{software}, in modo da evitare costose riprogettazioni e cambiamenti architetturali.
			\item \textbf{Implementazione e comprensione di \textit{test} prestazionali}. Capire il modo con cui sono implementati i \textit{test} di questo tipo è importante per comprendere con efficacia i confronti tra diverse librerie e soluzioni proposte \textit{online}. 
		\end{itemize}	
	
\section{Mancanze nell'insegnamento accademico}
	Complessivamente, uno \textit{stage} dovrebbe fornire allo stagista un bagaglio di conoscenze nuove che normalmente non possono essere insegnate in ambito scolastico. Tuttavia, l'università dovrebbe dare le nozioni di base per poter svolgere con profitto lo \textit{stage}, senza la necessità di apprendere in ambiente lavorativo determinati argomenti. A seguire includo una lista delle conoscenze che, secondo la mia esperienza, dovrebbero essere inserite nell'insegnamento accademico.
	\begin{itemize}
		\item \textbf{Utilizzo di IDE e sistemi di versionamento del codice}. L'attuale assetto didattico non prevede alcuna introduzione all'utilizzo di IDE, uno strumento fondamentale per un programmatore. La prima occasione dove lo studente si trova ad utilizzare seriamente un IDE è durante il progetto di Ingegneria del \textit{software}, collocato alla fine del percorso di studi. Se il gruppo non approfondisce l'utilizzo dell'IDE, lo studente rischia di trovarsi spiazzato durante i primi giorni dello \textit{stage}. Stesso discorso vale per il versionamento del codice, fondamentale quando si lavora in gruppo.
			\begin{itemize}[leftmargin=*, labelsep=0pt, leftmargin=0pt]
				\item[] \textbf{Soluzione proposta:} attualmente le lezioni di laboratorio, specialmente nei primi due anni, sono poco guidate. Sfruttare alcune di queste lezioni, soprattutto durante i corsi di Programmazione e di Programmazione ad oggetti, per insegnare allo studente un utilizzo maturo di IDE e versionamento del codice potrebbe essere una buona soluzione.
			\end{itemize}
		\item \textbf{Inadeguatezza nell'insegnamento di tecnologie \textit{web}}. Lo studente che inizia lo \textit{stage} ha una conoscenza troppo basilare dell'ambito \textit{web}, in quanto l'omonimo corso del terzo anno tratta moltissime tecnologie in modo molto superficiale. Inoltre, il \textit{focus} sull'accessibilità concorre a ridurre ulteriormente il tempo a disposizione di altri argomenti. Alcune delle nozioni fondamentali che secondo me dovrebbero essere fornite sono il funzionamento di AJAX e dei servizi REST.
		\begin{itemize}[leftmargin=*, labelsep=0pt, leftmargin=0pt]
			\item[] \textbf{Soluzione proposta:}  una soluzione accettabile potrebbe essere trasferire parte delle tecnologie insegnate dal corso di Tecnologie \textit{web} al corso di Basi di dati, in modo da poter approfondire i temi che ho appena citato. Personalmente ripristinerei l'assetto didattico in vigore fino a tre anni fa, che prevedeva l'inclusione di argomenti come HTML e PHP nel corso di Basi. Tuttavia, per renderla una soluzione davvero efficace, sarebbe necessario eliminare la ridondanza presente a quei tempi, durante i quali il corso di Tecnologie \textit{web} riprendeva gli stessi argomenti (in ambito \textit{web}) trattati nel corso di Basi.
		\end{itemize}
		\item \textbf{Educazione alle tecnologie}. Il corso di studi vede gli studenti impegnati nei primi due anni a studiare tecnologie classiche e ormai ben affermate. Durante il progetto di Ingegneria del \textit{software} (e in alcuni \textit{stage}) viene richiesta la scelta dello \textit{stack} tecnologico, e nella maggior parte dei casi le alternative comprendono tecnologie innovative. Ritengo che gli strumenti a disposizione dello studente siano del tutto insufficienti per effettuare una scelta matura e consapevole. Se lo \textit{stack} tecnologico non è fissato, lo studente si trova a dover effettuare una scelta che, molto spesso, viene fatta in completa casualità o seguendo i consigli, a volte non oggettivi, presenti \textit{online}. Dato che non è possibile illustrare tutte le tecnologie esistenti, sarebbe utile dare come base un insieme di criteri che permettano allo studente di prendere decisioni più consapevoli in ambito tecnologico.
		\begin{itemize}[leftmargin=*, labelsep=0pt, leftmargin=0pt]
			\item[] \textbf{Soluzione proposta:} i seminari tenuti durante il corso di Ingegneria del \textit{software} sono un buon punto di partenza, ma il \textit{focus} di ogni seminario è una singola tecnologia, difficile da collocare nello \textit{stack} tecnologico per studenti inesperti. Inoltre, essi vengono tenuti durante uno dei periodi più impegnativi del percorso accademico, quindi la partecipazione non è sempre garantita.
			
			Una buona soluzione potrebbe essere tenere dei seminari nel corso del secondo anno che diano criteri di base sulla scelta dello \textit{stack} tecnologico e informazioni sullo stato dell'arte raggiunto in vari campi dell'informatica.
		\end{itemize}
		
	\end{itemize}
	
	